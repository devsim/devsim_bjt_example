\documentclass[11pt]{article}
\newcommand{\devsim}{\mbox{DEVSIM}}
\newcommand{\devsimllc}{\mbox{DEVSIM LLC}}
\newcommand{\gmsh}{\mbox{Gmsh}}
\newcommand{\matplotlib}{\mbox{matplotlib}}
\newcommand{\numpy}{\mbox{NumPy}}
\newcommand{\python}{\mbox{Python}}
\newcommand{\visit}{\mbox{VisIt}}

\usepackage{indentfirst}
\usepackage{fullpage}
\usepackage{helvet}
\renewcommand{\familydefault}{\sfdefault}
\usepackage{mathptmx}
%\usepackage[dvips,colorlinks,breaklinks,pagebackref,citecolor=blue,urlcolor=blue,bookmarks,bookmarksnumbered,linkcolor=blue]{hyperref}
\usepackage[colorlinks,breaklinks,pagebackref,citecolor=blue,urlcolor=blue,bookmarks,bookmarksnumbered,linkcolor=blue]{hyperref}
\title{\devsim\ BJT Example}
\author{Juan E. Sanchez\\ juan.sanchez@devsim.com}

\setcounter{secnumdepth}{2}

\begin{document}
\maketitle
\thispagestyle{empty}
\section{Introduction}
This package includes the examples for the article ``Semiconductor Device Simulation Using \devsim''.  In this example, the meshing, modeling, simulation, and visualization for a bipolar junction transistor~(BJT) is developed.
%\section{Installation}
%\subsection{\devsim}
\devsim\ is an open source simulation software for technology computer-aided design~(TCAD) and is developed by \devsimllc.  It uses a generalized partial-differential equation~(PDE) approach to perform semiconductor device simulation on a mesh.  The software and documentation is available from \url{http://www.devsim.org}.

In addition to \devsim, the following software packages are used to perform meshing, analysis, and visualization of results.

~\\~\\
\noindent
\begin{tabular}{p{1.5cm}p{5cm}p{5cm}p{2cm}}
\hline\noalign{\smallskip}
Name & Description & Website & License$^a$  \\
\gmsh & Mesh Generator & \url{http://geuz.org/gmsh} & GPL\\
\matplotlib &  Python 2D Plotting Library & \url{http://matplotlib.org} & \matplotlib \\
\numpy &  Python Scientific Computing & \url{http://numpy.org} & BSD\\
\python &  Scripting Language & \url{http://python.org} & PSF\\
\visit &  Visualization Tool & \url{http://visit.llnl.gov} & BSD\\
\noalign{\smallskip}\hline\noalign{\smallskip}
\end{tabular}

\section{Running the Examples}
\begin{minipage}{\textwidth}
Here are some of the files in the package used for simulation.

\begin{tabular}{ll}
\texttt{bjt.geo} & Mesh description for \gmsh\\
\texttt{bjt.msh} & Resulting gmsh mesh\\
\texttt{initial\_guess.py}  & Creates initial guess from Potential only simulation\\
\texttt{refinement.py}  & Sets up E-field based refinements for creating background mesh\\
\texttt{netdoping.py}  & Specifies analytical doping profile\\
\texttt{bjt\_refine.py} & Runs \devsim\ to create a background mesh\\
\texttt{bjt\_bgmesh.pos} & Background mesh generated by \devsim\ for refinement using \gmsh\\
\texttt{physics/} & subdirectory containing physics files used in simulation.
\end{tabular}
\end{minipage}

\subsection{Meshing and Refinement}

The file \texttt{bjt.geo} contains the initial mesh specification for the bjt structure.  This file is run through \gmsh\ in order to create a triangular mesh for use in \devsim.  The resulting mesh file is called \texttt{bjt.msh}.  In order to create a mesh suitable for devsim, the \texttt{bjt\_refine.py} script is run to create a background mesh with element sizes appropriate for simulation.  The background mesh is then used with the original mesh specification to create a refined mesh.  This procedure is repeated until the mesh is sufficiently refined for use in \devsim.

\begin{minipage}{\textwidth}
The steps are:
\begin{verbatim}
gmsh -2 bjt.geo 
devsim bjt_refine.py 
gmsh -2 bjt.geo -bgm ./bjt_bgmesh.pos 
devsim bjt_refine.py 
gmsh -2 bjt.geo -bgm ./bjt_bgmesh.pos 
devsim bjt_refine.py 
gmsh -2 bjt.geo -bgm ./bjt_bgmesh.pos 
devsim bjt_refine.py 
gmsh -2 bjt.geo -bgm ./bjt_bgmesh.pos 
devsim bjt_refine.py 
\end{verbatim}
\end{minipage}
\\~\\

The resulting mesh from each \devsim\ run can be visualized by running Visit.
\begin{verbatim}
visit bjt_refine.tec
\end{verbatim}
\subsection{Simulation}
The dc and ac sweeps used in the publication are listed in \texttt{simsbatch.txt}.  These simulations can be run in sequence or in parallel.

\subsubsection{$V_c$ sweep}
For a given value of $V_b$, sweep $V_c$ from $0$ to $1.5$~V.
\begin{verbatim}
bjt_circuit2.py 0.1 &> data/vb2_0.1.out
\end{verbatim}

\subsubsection{$V_b$ sweep}
For a given value of $V_c$, sweep $V_b$ from $0$ to $1.0$~V.
\begin{verbatim}
devsim bjt_circuit3.py 0.0 &> data/vc_0.0.out
\end{verbatim}

\subsubsection{$V_e$ sweep}
For a given value of $V_c$, sweep $V_e$ from $0$ to $-1.0$~V.
\begin{verbatim}
devsim bjt_circuit4.py 0.0 &> data/ve_0.0.out
\end{verbatim}

\subsubsection{Small-signal ac sweep}
For a given value of $V_c$, sweep $V_e$ from $0$ to $-1.0$~V.  Do a small signal frequency sweep from fmin to fmax with given points per decade.
\begin{verbatim}
devsim bjt_circuit5.py 0.0 1e3 1e11 3 &> data/ssac_0.0.out
\end{verbatim}

\subsection{Visualization}
The \texttt{data/} directory contains scripts used to generate the plots used for publication.  A script was written to collect the data from the simulations to create plots using \texttt{matplotlib}.  This script is in the \texttt{data/} directory and is called \texttt{prep.sh}.  

\begin{tabular}{ll}
\texttt{ft.py} & Small-signal ft simulation\\
\texttt{gummel.py} & Ic, Ib versus Vbe.\\
\texttt{ic\_vec.py} & Ic versus Vce.\\
\end{tabular}
\end{document}

